\documentclass{beamer}
\usetheme{CambridgeUS}
\usepackage{listings}
\usepackage{blkarray}
\usepackage{listings}
\usepackage{subcaption}
\usepackage{url}
\usepackage{tikz}
\usepackage{tkz-euclide} % loads  TikZ and tkz-base
%\usetkzobj{all}
\usetikzlibrary{calc,math}
\usepackage{float}
\renewcommand{\vec}[1]{\mathbf{#1}}
\usepackage[export]{adjustbox}
\usepackage[utf8]{inputenc}
\usepackage{amsmath}
\usepackage{amsfonts}
\usepackage{tikz}
\usepackage{hyperref}
\usepackage{bm}
\usetikzlibrary{automata, positioning}
\providecommand{\pr}[1]{\ensuremath{\Pr\left(#1\right)}}
\providecommand{\mbf}{\mathbf}
\providecommand{\qfunc}[1]{\ensuremath{Q\left(#1\right)}}
\providecommand{\sbrak}[1]{\ensuremath{{}\left[#1\right]}}
\providecommand{\lsbrak}[1]{\ensuremath{{}\left[#1\right.}}
\providecommand{\rsbrak}[1]{\ensuremath{{}\left.#1\right]}}
\providecommand{\brak}[1]{\ensuremath{\left(#1\right)}}
\providecommand{\lbrak}[1]{\ensuremath{\left(#1\right.}}
\providecommand{\rbrak}[1]{\ensuremath{\left.#1\right)}}
\providecommand{\cbrak}[1]{\ensuremath{\left\{#1\right\}}}
\providecommand{\lcbrak}[1]{\ensuremath{\left\{#1\right.}}
\providecommand{\rcbrak}[1]{\ensuremath{\left.#1\right\}}}
\providecommand{\abs}[1]{\vert#1\vert}

\newcounter{saveenumi}
\newcommand{\seti}{\setcounter{saveenumi}{\value{enumi}}}
\newcommand{\conti}{\setcounter{enumi}{\value{saveenumi}}}
\usepackage{amsmath}
\setbeamertemplate{caption}[numbered]{}                               
                               

\title{AI1110 Assignment 6}
\author{DEEPSHIKHA-CS21BTECH11016}
\date{\today}
\logo{\large \LaTeX{}}


\begin{document}
\begin{frame}
		\titlepage
	\end{frame}

\begin{frame}{Outline}
  \tableofcontents
\end{frame}
\section{Abstract}
	\begin{frame}{Abstract}
		\begin{itemize}
			\item 	This document contains the solution to Question of Chapter 2 of Papoulis book.
		\end{itemize}
	\end{frame}
	
	
	\section{Question}
	\begin{frame}{Question}
		\begin{block}{\textbf{ Ex 2.13}}
		A box contains white and black balls. When two balls are drawn without replacement,
suppose the probability that both are white is 1/3.
	
	Find
	\begin{enumerate}
	\item Find the smallest number of balls in the box.
	\item How small can the total number of balls be if black balls are even in number? 
	\end{enumerate}	 
		\end{block}
	
	\end{frame}
	
\section{Theory}
\begin{frame}{Theory}
Let a = Number of white balls in the box.


Let b = Number of black balls in the box.


Let $W_k$ = "a white ball is drawn at the kth draw" .
\end{frame}

\section{Solution}
\begin{frame}{Solution}


We are given that \pr{W_1 W_2} = 1/3.
\begin{align}
\pr{W_1 W_2}=\pr{W_2 W_1}&=\pr{W_2|W_1}\pr{W_1}\\
    \frac{1}{3}&=\frac{a-1}{a+b-1}.\frac{a}{a+b}\\
    \frac{a}{a+b} &< \frac{a-1}{a+b-1}
\end{align}
From equation (2) and (3), we can rewrite as,
\begin{align}
    \brak{\frac{a-1}{a+b-1}}^2 < \frac{1}{3} < \brak{\frac{a}{a+b}}^2
\end{align}
This gives the inequalities,
\begin{align}
    (\sqrt{3} + 1)b/2 < a < 1 + (\sqrt{3} + 1)b/2 
\end{align}
\end{frame}


\begin{frame}{}
\begin{enumerate}
    \item 
    For b = 1, this gives 1.36 $<$ a $<$ 2.36, or a = 2, and we get ,
\begin{align}
    \pr{W_2 W_1}&=\frac{2}{3}.\frac{1}{2}\\
                &=\frac{1}{3}
\end{align}
Thus the smallest number of balls required is 3.
\end{enumerate}
\end{frame}



\begin{frame}{}
\begin{enumerate}
\item 
For b=even number, we can use equation (4), with b = 2, 4, ... as shown in Table 1. 
From the table,  10 is the smallest number of balls (a = 6, b = 4) that 
gives the desired probability. 
    \end{enumerate}
\end{frame}

\section{Table}
\begin{frame}{Table}
\begin{table}[ht!]
    \centering
    \begin{tabular}{|c|c|c|c|c|}
    \hline
     b   &  a & \pr{W_2 W_1}\\
    \hline\hline
    2   &  3 & $\frac{3}{4}.\frac{2}{4}=\frac{3}{10}\neq\frac{1}{3}$\\
    \hline
    4   &  6 & $\frac{6}{10}.\frac{5}{4}=\frac{1}{3}$\\
    \hline
    \end{tabular}
    \caption{Probability}
    \label{tab:my_label}
\end{table}
\end{frame}

    

\end{document}